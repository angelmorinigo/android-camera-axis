\section{Communication avec la caméra}
\subsection{Envoi de requêtes HTTP}
Afin de communiquer avec la caméra, Axis met à disposition des développeurs une API nommée \textit{VAPIX\footnote{\label{vapix}
http://www.axis.com/techsup/cam_servers/dev/cam\_http\_api\_index.php}}. Cette interface est basée sur le protocole \textit{HTTP}, permettant un accès aux fonctionnalités de la caméra par de simples URLs.
Par exemple, pour lister les paramètres réseau de la caméra nous utilisons l'URL suivante :
\begin{lstlisting}
http://myserver/axis-cgi/admin/param.cgi?action=list&group=Network
\end{lstlisting}
En cas de réussite de cette requête, nous recevons la réponse suivante :
\begin{lstlisting}
HTTP/1.0 200 OK\r\n
Content-Type: text/plain\n
\n
root.Network.IPAddress=<adresse ip>\n
root.Network.SubnetMask=<masque reseau>\n
\end{lstlisting}

Certaines requêtes comme le contrôle PTZ de la caméra ne provoquent pas de réponse (code HTTP 200) mais indiquent simplement que la caméra a bien reçu la requête (code HTTP 204), cepedant nous ne sommes pas informés de la réalisation de l'action demandée.
Nous avons implémenté le mécanisme de requêtes \textit{HTTP} vers la caméra à l'aide de l'objet \textit{HttpURLConnection} dans la méthode sendCommand de la classe CameraControl.