\documentclass[a4paper,10pt]{report}

\usepackage[utf8]{inputenc}
\usepackage{ulem}
\usepackage[frenchb]{babel}
\usepackage{graphicx}
\usepackage[final]{pdfpages}
\usepackage{url}
\usepackage{xcolor}
\usepackage{listings}
\lstset{language=java}

\begin{document}
\author{Jerome NAHELOU, Quentin NEBOUT, Romain SOLVE, Fabien QUINTARD}

\chapter{Analyse des besoins}
Besoins Fonctionnels

- Retransmettre le flux vidéo d'une manière fluide
- Afficher une ou plusieurs vues de caméras
- Gérer une liste de caméras (ajout, modification, suppression)
- Importer/Exporter la liste de caméras
- Contrôler la caméra tactilement et utiliser les fonctionnalités spécifiques à celle-ci
- S'adapter au réseau disponible Wifi/3G
- Pouvoir faire une capture d'écran et l'enregistrer
- Permettre la détection de mouvement et la mise en tâche de fond de plusieurs fenêtres
- Permettre de régler les propriétés générales de l'application et les propriétés de la détection de mouvement
- Envoyer des notifications pour prévenir du snapshot / mouvement détecté 
- Récupérer des informations sur une caméra à partir d'un QrCode

Besoins Non Fonctionnels

- Avoir de bonnes performances, avec une vitesse de rafraichissement correcte
- Être sûre lors de l'exécution en cas de faible connectivité
- Avoir une vitesse de réponse faible avec un contrôle de la caméra réactif
- Sauvegarder les données sans risques
- Avoir une bonne ergonomie et être facile d'utilisation
- Etre réactif à la détection de mouvement
- Être comptatible avec le modèle de caméra (fonctionnalités variables)

\end{document}
