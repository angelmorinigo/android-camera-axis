\section{Communication avec la caméra}
\subsection{Envoi de requêtes HTTP}
Afin de communiquer avec la caméra, Axis met à disposition des développeurs une
API nommée \textit{VAPIX\footnote{\label{vapix}
http://www.axis.com/techsup/cam\_servers/dev/cam\_http\_api\_index.php}}. Cette
interface est basée sur le protocole HTTP, permettant un accès aux fonctionnalités de la caméra par de simples URLs. Par exemple, pour lister les paramètres réseau de la caméra nous utilisons l'URL suivante :
\begin{lstlisting}
http://myserver/axis-cgi/admin/param.cgi?action=list&group=Network
\end{lstlisting}
En cas de réussite de la requête, nous recevons la réponse suivante :
\begin{lstlisting}
HTTP/1.0 200 OK\r\n
Content-Type: text/plain\n
\n
root.Network.IPAddress=<adresse ip>\n
root.Network.SubnetMask=<masque reseau>\n
\end{lstlisting}

Certaines requêtes comme le contrôle PTZ de la caméra ne provoquent pas de réponse (code HTTP 200) mais indiquent simplement que la requête a bien été reçue (code HTTP 204), cependant nous ne sommes pas informés de la réalisation de l'action demandée.
Nous avons implémenté le mécanisme de requêtes \textit{HTTP} vers la caméra à l'aide de l'objet \textit{HttpURLConnection} dans la méthode \textit{sendCommand} de la classe \textit{CameraControl}. La commande passée en paramètre est concaténée à l'adresse IP de la caméra pour construire l'URL visée.
Nous initialisons alors une connexion HTTP vers l'URL avant d'ajouter un
\textit{timeout} au bout duquel la requête est considérée comme un échec, cette
valeur étant paramétrable par l'utilisateur dans les préférences de l'application. Le paquet est également forgé avec le champ \textit{Authorization} qui englobe les identifiants encodés en \textit{base64} pour l'authentification :
\begin{lstlisting}
con.setRequestProperty("Authorization",
		base64Encoder.userNamePasswordBase64(cam.login, cam.pass));
\end{lstlisting}
Nous avons pour cela réutilisé la classe existante
\textit{base64Encoder}\footnote{\label{base64}http://www.rgagnon.com/javadetails/java-0084.html}
et ajouté une méthode qui renvoie les identifiants en \textit{base64} sous la forme \textit{login:password}. Par conséquent, l'authentification est réalisée à chaque requête, même si toutes les actions ne nécessitent pas de droits spécifiques. La méthode retourne enfin l'objet \textit{HttpURLConnection} qui pourra être utilisé pour analyser la réponse et récupérer le contenu avec \textit{getInputStream} dans le cas d'une capture par exemple.

\subsection{Chargement de la configuration}
Avant de récupérer le flux vidéo distant, nous utilisons la méthode \textit{loadConfig} dans le constructeur de la classe \textit{CameraControl} afin de récupérer et établir la liste des fonctionnalités supportées par la caméra. Cette méthode permet aussi de récupérer certains paramètres comme les résolutions et formats d'image disponibles.
Le principe est d'envoyer deux requêtes à la caméra puis de parser le contenu des réponse pour remplir les deux tableaux d'entiers suivants :
\begin{itemize}
	\item \textit{functionProperties} qui conserve les différentes propriétés des fonctions \textit{PAN}, \textit{TILT}, \textit{ZOOM}, \textit{FOCUS} et \textit{IRIS}.
	\item \textit{currentConfig} qui conserve l'état actuel des fonctionnalités
	représenté par les constantes \textit{NOT\_SUPPORTED}, \textit{DISABLED} ou \textit{ENABLED}.
\end{itemize}
La première requête est envoyée à l'adresse suivante :
\begin{lstlisting}
http://myserver/axis-cgi/com/ptz.cgi?info=1
\end{lstlisting}
Le résultat est un long message texte avec pour chaque ligne une chaîne au format \textit{variable=valeur}.
Il suffit alors de parser ligne par ligne en isolant le mot \textit{variable} et en le comparant avec différentes occurences.
Par exemple, le code suivant récupère les capacités de la caméra pour la fonctionnalité "pan" :
\begin{lstlisting}
if (property.contains("pan")) {
	if (property.contentEquals("pan"))
		functionProperties[PAN] += ABSOLUTE;
	else if (property.contentEquals("rpan"))
		functionProperties[PAN] += RELATIVE;
	else if (property .contentEquals("continuouspantiltmove")) {
		functionProperties[PAN] += CONTINUOUS;
		functionProperties[TILT] += CONTINUOUS;
	}
}
\end{lstlisting}
Les constantes identifiant les différentes capacités possibles sont : \textit{ABSOLUTE}, \textit{RELATIVE}, \textit{DIGITAL}, \textit{AUTO}, \textit{CONTINUOUS}.
A la fin du parsage de la première requête, \textit{currentConfig} est mis à jour pour les fonctionnalités concernées par \textit{functionProperties} avec la boucle suivante :
\begin{lstlisting}
for (int i = 0; i < NB_BASIC_FUNC; i++)
	if (functionProperties[i] > 0)
		currentConfig[i] = ENABLED;
\end{lstlisting}

La deuxième requête est quant à elle envoyée à l'adresse suivante :
\begin{lstlisting}
http://myserver/axis-cgi/admin/param.cgi?action=list&group=Properties.Motion.Motion,Properties.Audio.Audio,Properties.Image
\end{lstlisting}
Le parsage de la réponse permet de mettre à jour la configuration relative à la détection de mouvements, à l'audio et à l'image (résolutions, rotations, formats vidéo).

Une fois le chargement de la configuration effectué, il sera possible de savoir si telle fonctionnalité est supportée et/ou activée par l'appel aux méthodes respectives \textit{isSupported} et \textit{isEnabled}.
De même, \textit{enableFunction} et \textit{disableFunction} permettront de faire varier l'état d'une fonction (\textit{ENABLED}/\textit{DISABLED}). Cependant nous n'avons pas utilisé ces deux dernières méthodes car il n'est pas possible de savoir dans quel état actuel se trouve la caméra pour une certaine fonctionnalité.
En effet, plusieurs personnes pouvant agir tour à tour sur les paramètres de la caméra.

\subsection{Liste des commandes utilisées}
Le tableau suivant présente les différentes commandes distantes que nous avons
utilisées dans notre application Android, associant l'adresse pointée (basée toujours sur l'adresse IP de la caméra) à l'action effectuée par la caméra. \begin{table}[H]
\centering
\begin{tabular}{|p{0.5\linewidth}|p{0.5\linewidth}|}
\hline
Partie variable URL & Action
\hline
axis-cgi/com/ptz.cgi?info=1 & Retourne la liste globale des capacités fonctionnelles de la caméra
\hline
axis-cgi/admin/param.cgi?action=list\\\&group=Properties... & Retourne l'état
actuel des paramètres spécifiés (Motion, Audio, Image, \dots)
\hline
axis-cgi/mjpg/video.cgi?resolution=<resolution> & Retourne le flux MJPEG de
résolution <resolution>
\hline
axis-cgi/jpg/image.cgi?resolution=<resolution>
& Retourne une capture d'écran de résolution <resolution>
\hline
axis-cgi/operator/param.cgi?action=add&group=Motion&template=motion & Ajoute une
fenêtre de détection de mouvements
\hline
axis-cgi/operator/param.cgi?action=remove&group=Motion.<groupeID> & Supprime une
fenêtre de détection de mouvements existante d'identifiant <groupeID>
\hline
axis-cgi/motion/motiondata.cgi?group=<groupeID> & Retourne à intervalles
réguliers les niveaux de détection de la fenêtre d'identifiant <groupeID>
\hline
axis-cgi/operator/param.cgi?action=update&Motion.M<groupeID>.<param>=<valeur> &
Met à jour les paramètres de la fenêtre de détection d'identifiant <groupeID>
\hline
axis-cgi/operator/param.cgi?action=update&Motion.M<groupeID>.<param>=<valeur> &
Met à jour les paramètres de la fenêtre de détection d'identifiant <groupeID>
\hline
axis-cgi/com/ptz.cgi?<param>=<valeur>
\begin{itemize}
  \item <param> parmi { rpan, rtilt, rzoom, riris, rfocus, rbrightness }
  \item <valeur> une valeur numérique
\end{itemize} &
Effectue l'action <param> correspondante (pan, tilt, zoom, iris, focus) en
appliquant <valeur> de manière relative (calculé à partir de la position actuelle)
\hline
axis-cgi/com/ptz.cgi?<param>=<valeur>
\begin{itemize}
  \item <param> une valeur parmi { autofocus, autoiris, ircutfilter, backlight }
  \item <valeur> une valeur parmi { on, off, auto}
\end{itemize} &
Active/désactive ou met en mode automatique la fonctionnalité <param>
\hline
\end{tabular}
\caption{Liste des commandes HTTP}
\end{table}