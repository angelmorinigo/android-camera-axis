La communication pour le contrôle du Pan/Tilt/Zoom et des fonctionnalités spécifiques (snapshot, contrôles avancés) de la caméra s'effectuent à travers la classe CameraControl.
Les fonctions changeValFunc et switchAutoFunc permettent d'envoyer les requêtes HTTP à la caméra pour changer les valeurs associées aux paramètres :
- PAN, TILT, FOCUS, IRIS, BRIGHTNESS qui prennent une valeur flottante
- AUTOFOCUS, AUTOIRIS, AUTO_IR, BACKLIGHT dont la valeur est comprise dans { on, off, auto }
Une réponse de code HttpURLConnection.HTTP_NO_CONTENT (204) indique la réussite de la requête.

La fonction takeSnapshot réalise la requête de capture d'écran avec une résolution dont la valeur est passée en paramètre.
Elle retourne les données renvoyées par la caméra sous forme d'un objet Bitmap.


Cette classe s'occupe également des activation/désactivation de la détection de mouvements.